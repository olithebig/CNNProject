\chapter{Einleitung}
Im Rahmen des Wettbewerbes „CLOUD-MEISTER 2017: DER WAGO - IDEENWETTBEWERB“ hat ein Team von Studierenden um Herrn Reich den Weinschorle-Automaten „WISH 4.0“ entwickelt. Am Smartphone, Tablet oder Computer lässt sich dabei das gewünschte Mischverhältnis des Schorles per Webseite angeben, das eingestellte Glas wird per Webcam bemessen, das Volumen bestimmt und anschließend über mehrere Ventile das Schorle gemischt.

Außerdem kann der Automat auch Barcodes erkennen, um früher festgelegte Mischungen wieder zu erkennen. Im Zentrum der Umsetzung standen dabei die in Tabelle \ref{tab:Umsetzung_Industrie_4_0} dargestellten Industrie 4.0 Konzepte.

\begin{table}
	\caption{Umsetzung der Industrie 4.0 Konzepte im WISH 4.0}
	\label{tab:Umsetzung_Industrie_4_0}
	\begin{tabular}{p{0.5\linewidth}|p{0.5\linewidth}}
		Industrie 4.0 Konzept & WISH 4.0 Umsetzung \\
		\hline \hline
		Individualisierte Produktion & Individuelles Glas, Mischverhältnis, Füllmenge und Glas-Label \\
		\hline
		Prozessoptimierung in der Cloud & Nutzungsabhängige Auswahl des Befüllers \\
		\hline
		Zustandsüberwachung (Condition Monitoring) in der Cloud & Parameter der Befüller wird überwacht \\
		\hline
		Optimierte Wartung & Je nach Nutzung wird der Bedarf einer Wartung bestimmt \\
		\hline
		Cloud Steuerung & Meta-Steuerbefehle (z.B. fülle Glas) werden von der Cloud an die Lokale Steuerung geschickt
		Lokale Steuerung (WAGO-Controller) führt die Geräte-Befehle aus \\
		\hline
		Visual Quality Management & Visuelle Erkennung von verschiedenen Gläsern in der Cloud \\
		\hline
		Smart Logistics  & Erkennung des Barcodes, Kupplung zu Glas, und Speicherung für weitere Verarbeitung \\
		\hline
		Supply Chain Compliance & Einhaltung vordefinierter Compliance Vorgaben bezüglich des zu druckenden Labels \\
	\end{tabular}
\end{table}