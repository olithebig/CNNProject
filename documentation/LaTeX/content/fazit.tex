\chapter{Fazit}
Das Projekt ist abgeschlossen, ist aufgrund der nicht implementierten Volumenberechnung als funktionale Anforderung als kritisch einzustufen.

Bezogen auf unsere Planung waren unsere Arbeitspakete unproblematisch, da diese unabhängig voneinander waren, weshalb unser kritischer Pfad wesentlich kürzer als die vorgesehene Projektdauer war. Diese haben wir je-doch voll ausnutzen müssen, da wir zum parallel an verschiedenen Anforderungen nicht genug Teilnehmer waren.

\section{Ausblick}
Angesichts der in unserem Projekt nicht erfüllten funktionalen Anforderung, die Volumenberechnung der Gläser, wäre eine Weiterentwicklung des Automaten in dieser Richtung durchaus denkbar. Da sich die Kinect aus oben genannten Bedingungen dazu jedoch nicht eignet, müsste man sich auf diesem Gebiet nach passenderen Technologien umsehen.

Weiterhin könnte man sich mit der Analyse weiterer Komponenten beschäftigen, wie zum Beispiel mit dem Füllstand der Silos. Dazu könnte man die Befülltürme für einen Überbau vorbereiten (Obere Plattformen aktuell noch lo-se). Dies würde neue Möglichkeiten zur Analyse des Zustandes vom Automaten bieten, man könnte zum Beispiel den kompletten Drehteller überwachen, und nicht nur die Entnahmeposition der Gläser, wie es aktuell implementiert ist.

Eine weitere praktische Ergänzung wäre ein direkt im Automaten integrierter Label-Drucker, welcher nach Abschluss einer Bestellung in der Webanwendung direkt das zugehörige Label bereitstellt.

Auf Software-Ebene wäre wichtig das Frontend (die Webanwendung) an die neue Routine anzupassen. Mit der Erkennung des Verschmutzungsgrades haben wir einen neuen möglichen Zustand definiert, der vom Automaten ein-genommen werden kann, welcher auf der Status-Seite der Webanwendung jedoch noch nicht berücksichtigt wird. Die gilt auch für eventuelle andere Weiterentwicklungen, wie bereits oben angemerkt.
